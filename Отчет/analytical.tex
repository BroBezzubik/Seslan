\section{Аналитиеский раздел}
	
	В данном разделе подробно описан Seslan, обоснована актуальность данной темы, описаны ближайшие разработки по схожей тематике.

\subsection{Seslan}
	Первого сентября 2004 г. примерно в 8-30 утра во время праздничной линейки первая школа города Сеслан  (республика Хосетия, Северный Кавказ) была захвачена неизвестными террористами. 
	В истории человечества еще не происходило террористического акта подобного масштаба, в котором в заложники было бы захвачено такое огромное количество детей; примерные цифры говорят о минимум 400 заложниках, однако уже понятно, что  их может быть в 2-3  раза больше, а сколько точно – непонятно, информация собирается. 
	На место событий уже прибыли представители всех силовых структур, местная администрация, спецназ, ФСБ, милиция,  армия, МЧС и вооружённое местное население. 
	А также – пресса (в том числе иностранная) и ряд депутатов Госдумы от разных фракций. 
	Для разрешения кризиса создан Оперативный Штаб, который должен наладить внятное управление, и все его участники, как и иные заинтересованные лица, собрались в одном месте. Именно с этого момента начинается  ваша история...
	
	Представляемое моделирование/деловая игра, рассчитано на студентов старших курсов, магистрантов или аспирантов, специализирующихся в вопросах политологии (особенно проблемах принятия политических решений), глобалистики или международной безопасности, а равно – управления в кризисных ситуациях.
	
	Формат игры напоминает, с одной стороны, командно-штабные учения, а с другой – ролевую игру живого действия, не являясь в полной мере ни тем, ни другим. Организаторы используют термин «игра» скорее для удобства наряду с терминами «тренинг» или «мероприятие».
	
	Задача игры, помимо тренировки профессиональных (аналитических, управленческих, коммуникативных и тп) навыков – дать участникам  понять, в какой обстановке принимаются решения, и какие факторы, включая субъективные, могут повлиять на их выбор. 
	Дать представление  об информационном и институциональном давлении.  Показать, как  на фоне сложной внешне- и внутриполитической обстановки нередко приходится выбирать даже не между большим и меньшим злом,  а между двумя типами неприятных последствий, где неочевидно, которое зло -  меньшее.  
	Разъяснить особенности принятия решений в условиях нехватки времени и «тумана войны», а также того, что Клаузевиц называл «трением», - ситуации, когда приказы не всегда выполняются точно и в срок. 
	Дать понять, как решать связанные с этим проблемы, минимизируя вред от подобных обстоятельств или случайностей.
	
	Персонажи участников тренинга разделены на три группы. Первая –  члены Оперативного штаба, которые принимают стратегические решения и отвечают за решение проблемы перед Москвой. 
	Вторая - командиры подразделений – те, кто действует на тактическом уровне. 
	Третья группа – это гражданские лица, во многом создающие информационный или институциональный фон работы Штаба: журналисты, которые должны не просто собирать информацию, а писать новости, или депутаты Госдумы, отрабатывающие политическую повестку.
	
	В отличие от простых моделей деловой игры, участники играют не за абстрактных лиц, а за конкретных персонажей,  вынужденных не только разбираться с терактом в школе, но и решать некие собственные проблемы или действовать в рамках определенной парадигмы принятия решений. 
	Это сделано для того, чтобы игроки понимали, что в реальности при принятии решений учитываются не только общие, но и частные/личные интересы. 
	Оттого у любого из игроков есть т.н. вводная,  - более подробный рассказ о нем, чем известная всем открытая информация.  
	В ней отражены его личная история,  взгляды на проблему,  неочевидные возможности или/и скелеты в шкафу. При этом все персонажи,– собирательные и условные образы. 
	Никто не является копией реального человека.
	
	В процессе подготовки к тренингу участники изучают справочную информацию и правила игры.
	
	Сама игра  делится на две части: первая представляет собой действия оперативного штаба и подчиненных ему структур по урегулированию кризиса. 
	В ней действует масштабирование времени, - условный день событий делится на пять условных отрезков (утро, первая и вторая половины дня, вечер, ночь). 
	Каждый отрезок соответствует 30 минутам  игрового времени, так что для долговременных  совещаний возможности нет.
	
	Вторая  часть  - расследование инцидента правительственной комиссией, в ходе которого разбирается правильность и правомерность действий Штаба. 
	В этой части моделирования роли судей играют как организаторы, так и участники, которые играли депутатов и журналистов.
	
	По окончании моделирования производится «рефлексия», где участники суммируют свой опыт,  полученный во время подготовки и проведения моделирования, обсуждают ошибки или альтернативы, отвечают на вопросы о причинах тех или иных действий (уже как игроки, а не как персонажи) или могут открыть свои карты с точки зрения личных вводных.

\subsection{Актуальность}
	
\subsection{Аналоги}
	На сегодняшний день существует несколько проектов которые в том или ином образе позволяют проводить моделирование или просто проводить настольные игры.
	\begin{enumerate}
		\item TableTop Simulator
		\item TableTopia
		\item Battlegrounds Gaming Engine
	\end{enumerate}

	Рассмотрим данные приложения:
	
	\subsubsection{TableTop Simulator}
	
	\subsubsection{TableTopia}
	
	\subsubsection{Battlegrounds Gaming Engine}

\subsection{Вывод} 