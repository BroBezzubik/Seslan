\newpage
\section{Технологический}
В данном разделе приводятся средства программной реализации, описываются развертывание, указываются формат данных, входных и выходных файлов, приводится пользовательский интерфейс
программы.

\subsection{Выбор средст программной реализации}
	
	Программный продукт реализован на python 3.6 с использование Django 3.8.5 [\url{https://www.djangoproject.com/}] и postrgreSQL[\url{https://www.postgresql.org/}] в качестве базы данных. PyCharm 2020.2.02 в качестве IDEI [\url{https://www.jetbrains.com/pycharm/download/#section=windows}]. Така я студент, мне была предоставленна полная версия.
	Данное компоновка обеспечивает легкий перенос на другую платформу. Удобство разработки и по причине опыта работы с Django.
	
\subsection{Развертывание}

	Для того что бы развернуть данную систему необходимо выполнить следующие шаги:
	\begin{enumerate}
		\item Установить и настроить python
		\begin{enumerate}
			\item Скачать и установить python 3.8.5;
			\item Установить pip последней версии;
			\item Установить все необходимые библиотеки из requirements.txt.
		\end{enumerate}
		\item Установить и настроить postgreSQL
		\begin{enumerate}
			\item Скачать и установить postgreSQL;
			\item Для удобства так же установить pgAdmin;
			\item Создать пользователя и пароль;
			\item Создать базу данных под названием $Seslan\_db$
		\end{enumerate}
		\item Обновить настройки для django
		\begin{enumerate}
			\item Выставить пароль и логин для подключения к базе данных.
			\item Выполнить команды:
			\begin{enumerate}
				\item python manage.py migrate
				\item python manage.py collectstatic
				\item python manage.py createsuperuser - с помощью данной команды создается запись админа.
			\end{enumerate}
			\item Python manage.py runserver
		\end{enumerate}
	\end{enumerate}

\subsection{Формат данных}

	Программа принимает на вход данные полученные от пользователя с клиентского интерфейса. Данные могут представлены в виде изображений, текста либо чисел. Для общения между клиентом и сервером используется формат данных под названием JSON.

	\begin{enumerate}
		\item 
	\end{enumerate}